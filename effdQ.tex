\documentclass{article}
\usepackage{graphicx}
\usepackage{color}

\title{Efficient $\Delta Q$}
\author{Geet Duggal}
\date{Spring 2007ish}
\begin{document}
\maketitle

One of our concerns is being able to calculate the 
modularity efficiently on each Monte Carlo move\footnote{This first revision
also includes a C implementation that calculates the modularity 
given an edge and partition file: \texttt{./Q graph.edg nodes.par}. The
partition file simply associates a node with a community ID in the same format
of the edge file.}.  Modularity can be
conveniently expressed as \cite{cnm-community}:
\begin{equation}
Q = \displaystyle\sum_{i \in C}(e_{ii}-a_i^2)
\end{equation}
where $C$ is the set of communities
This essentially amounts
to being able to efficiently keep track changes in these two variables:
\begin{equation}
\label{edgefrac}
e_{ii} = \frac{1}{2m}\displaystyle\sum_{vw}A_{vw}\delta(c_v, i)\delta(c_w, i)\\
\end{equation}
\begin{equation}
\label{rootprob}
a_i = \frac{1}{2m}\displaystyle\sum_vk_v\delta(c_v, i)
\end{equation}
reminding ourselves that $m$ is the number of edges in the undirected graph,
$c_v$ indicates vertex $v$ belongs to community $c_v$, and $A_{vw}$ is the
adjacency matrix.  Stated in simpler terms, modularity is capturing
the total fraction of within-community edges subtracted
from what we would expect from a random graph for that community 
\emph{for all} communities.

Given that changes in modularity ($\Delta Q$) in the Metropolis Monte Carlo 
scheme described in \cite{massen-doye-community} involve changing the community assigned
to a random node, the change in modularity is only caused by that node and
the two communities it affects.  Aside from the case where the node does
not change its community, we can explore how to efficiently recalculate 
the modularity from the previous value.  

Suppose the two communities in the comparison  are $i$ and $j$, 
where we move $v'$ from $i$ to $j$: 
\begin{enumerate}
\item the change in equation (\ref{edgefrac}) involves subtracting
      the number of edges 
      from $i$ associated with $v'$ ($E_{iv'}$) and
      adding the number of edges to $j$ that are now newly associated
      with $v'$, ($E_{jv'})$.
\item the change in equation (\ref{rootprob}) involves properly 
      subtracting $k_{v'}$ from community $i$'s square term and adding it 
      to community $j$'s square term.  
\end{enumerate}
Since item 1 has no ugly squares: 
\begin{eqnarray*}
  \Delta e = e_{ii}(t+1) - e_{ii}(t) = \frac{1}{m}(E_{jv'} - E_{iv'})
\end{eqnarray*} 
But for item 2, we have:
\begin{eqnarray*}
  a_i(t) = \frac{1}{2m}\displaystyle\sum_vk_v\delta(c_v, i)\\
  a_j(t) = \frac{1}{2m}\displaystyle\sum_vk_v\delta(c_v, j)\\
  a_i(t+1) = a_i(t) - \frac{1}{2m}k_{v'}\\
  a_j(t+1) = a_j(t) + \frac{1}{2m}k_{v'}\\
  \Delta a^2 = \left[[a_j(t+1)]^2+[a_i(t+1)]^2\right] - 
              \left[[a_j(t)]^2+[a_i(t)]^2\right]\\
            = \left[\left(a_j(t) + \frac{1}{2m}k_{v'}\right)^2+
                    \left(a_i(t) - \frac{1}{2m}k_{v'}\right)^2\right] -
                    \left[[(a_j(t)]^2+[a_i(t)]^2\right]\\
            = \left[\frac{1}{4m^2}k_{v'}^2+\frac{1}{m}a_j(t)k_{v'} +
                    \frac{1}{4m^2}k_{v'}^2-\frac{1}{m}a_i(t)k_{v'} \right]\\
            = \frac{k_{v'}}{m}
              \left(\frac{k_{v'}}{2m}+a_j(t)-a_i(t)\right)
\end{eqnarray*}

Thus, after some basic algebra, we can express our change in modularity
as a simple operation by only storing a few running sums in memory:
\begin{equation}
  \Delta Q = \frac{1}{m}\left[(E_{jv'} - E_{iv'}) - k_{v'}
             \left(\frac{k_{v'}}{2m}+a_j(t)-a_i(t)\right)\right]
\end{equation}
recalling that we are transferring $v'$ from community $i$ to $j$. 
$\diamond$

\bibliographystyle{plain}
\bibliography{rlib}
\end{document}
